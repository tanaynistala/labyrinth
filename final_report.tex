\documentclass{article}
\usepackage[utf8]{inputenc}

\title{TurtleBot Maze Solver}
\author{Tanay Nistala, Stephanie Wilson, Iris Yang}
\date{\vspace{-1em}}

\usepackage{natbib}
\usepackage{graphicx}

\begin{document}

\maketitle

% for final report only
%\begin{abstract}
%A one paragraph high level description of the work. The abstract is typically the first thing someone would read from the paper before deciding whether to continue reading and hence, serves as an advertisement to the reader to read the whole paper. 
%\end{abstract}

\section{Introduction}

%The introduction should describe the problem (in a non-technical manner, i.e., without math, equations, etc.), as well as motivate the problem, i.e., why is it important?

In the field of robotics, maze solution by a robot is an interesting challenge because of the localization and path finding algorithms used to perform it. This specific application of path finding and localization can be used for finding the best ways for a robot to navigate a space efficiently and safely, and has been used for the development of many path finding algorithms. Dynamic robot path finding that adapts to obstacles in surroundings can be used for many useful tasks for robots, like finding its way through unknown areas of people or difficult terrain. Robots that conduct tours, scout out areas unreachable by people, and map out terrain can all use algorithms like these to gather information more effectively. 


We propose to test several different established methods of localization and path finding of a Turtlebot simulation on different mazes that we generate. This will require fine-tuning of said methods to allow the Turtlebot to expertly navigate the mazes, and we will compare how fast they are able to localize and navigate the maze. Our results will provide more insight into how robots map out unknown areas, hopefully leading to future advancements in localization and path-finding algorithms. 



\section{Background Related Work}

%Following, you should provide the necessary background and discuss related work in the Prob Rob literature. This section should also be about a page. Citations should be in BibTeX format \citep{thrun2005probabilistic}.

We can talk about the references below here. 




\section{Technical Approach / Methodology / Theoretical Framework}

% for proposal
%Describe how you will approach the problem and its technical formulation. Feel free to re-state the basic RL formulas (e.g., if using Q-learning, state the update rule or the formula for what the Q function approximates). 

\subsection{SLAM}
SLAM, or \textbf{S}imultaneous \textbf{L}ocalization \textbf{a}nd \textbf{M}apping, allows robots to keep track of their surroundings as well as their specific location on the map. SLAM usually consists of two elements: a tool to observe the environment and measure around the distance from it, like LiDAR or sonar, and software that extracts and identifies the objects in the area that it is mapping out.  
\\
We are using the SLAM algorithms directly provided to us in ROS. ROS provides us with several different SLAM algorithm implementations, including: Gmapping, Hector, and Karto. The mathematical details of these are shown below: 
\\\\(insert math equations)
\\\\ Gmapping operates by using laser scan/odometry data to learn maps at runtime. It separates positioning and mapping processes and scans the area that is in front of it to map the area as it moves. 
\\
\indent Hector leverages both LiDAR and 2D pose estimates, uses odometry for mapping and localization. Hector will fail if the odometry fails, which makes it more failure prone than the other algorithms. Due to this unreliability of Hector, we will be using it during experiments, but not without caution. 
\\
\indent Karto uses odometry, but unlike Hector, doesn’t fail when odometry fails. This makes it a more reliable version of Hector, and a comparison area against Gmapping. 


% for final report
% A detailed description of your problem (with math, notation, algorithms, figures, etc.). Use footnotes to cite links to your code or videos\footnote{All developed source code for this project is available at ...}

\subsection{Notation and Problem Formulation}

How did you formalize the problem using mathematical notation?

\subsection{Tasks}

Subsections are useful for breaking down the problem into sub-parts. For example, you could break down the tasks for your project and list them one by one. 

% for final report
\section{Experimental Results / Technical Demonstration}

A description of how you evaluated or demonstrated your solution.\footnote{a video of the robot doing x y z is available at...} 

% for proposal
%\section{Experiments and Evaluation}

%Describe how you will evaluate your approach/solution. What constitutes success? What metrics will you use? Do you have any preliminary hypothesis that you plan to test? Also, describe the RL domain or environment you plan to use. 

% for final report
\section{Conclusion and Future Work}

A high level summary of what was accomplished, along with a discussion on limitations and avenues for future work (typically 2 to 3 paragraphs). 


%\section{Timeline and Individual Responsibilities}

%State the timeline in terms of weeks and milestones you want to achieve. If working on a team, state what the individual responsibilities are at this point (i.e., who is going to do what, these may of course change over the course of the project). \cite{short2010no}.

\bibliographystyle{plain}
\begin{thebibliography}{9}


\bibitem{Yarovoi2024}
A. Yarovoi and Y. K. Cho, ``Review of simultaneous localization and mapping (SLAM) for construction robotics applications,'' Automation in Construction, vol. 162, pp. 105344, 2024, doi: 10.1016/j.autcon.2024.105344.

\bibitem{Thale2020}
S. P. Thale, M. M. Prabhu, P. V. Thakur, P. Kadam, ``ROS based SLAM implementation for Autonomous navigation using Turtlebot,'' ITM Web Conf., vol. 32, pp. 01011, 2020, doi: 10.1051/itmconf/20203201011.

\bibitem{Abiyev2017}
R. H. Abiyev, M. Arslan, I. Gunsel, and A. Cagman, ``Robot Pathfinding Using Vision Based Obstacle Detection,'' in Proc. 2017 3rd IEEE International Conference on Cybernetics (CYBCONF), Exeter, UK, 2017, pp. 1-6, doi: 10.1109/CYBConf.2017.7985805.

\bibitem{Turnage2016}
D. M. Turnage, ``Simulation results for localization and mapping algorithms,'' in Proc. 2016 Winter Simulation Conference (WSC), Washington, DC, USA, 2016, pp. 3040-3051, doi: 10.1109/WSC.2016.7822338.

\bibitem{Cherubini2014}
A. Cherubini, F. Spindler, and F. Chaumette, ``Autonomous Visual Navigation and Laser-Based Moving Obstacle Avoidance,'' IEEE Transactions on Intelligent Transportation Systems, vol. 15, no. 5, pp. 2101-2110, Oct. 2014, doi: 10.1109/TITS.2014.2308977.

\bibitem{Sang2024}
W. Sang, Y. Yue, K. Zhai, and M. Lin,  ``Research on AGV Path Planning Integrating an Improved A* Algorithm and DWA Algorithm,'' Applied Sciences, vol. 14, no. 17, p. 7551, 2024, doi: 10.3390/app14177551.

\bibitem{Lai2023}
X. Lai, et al., ``Enhanced DWA Algorithm for Local Path Planning of Mobile Robot,'' The Industrial Robot, vol. 50, no. 1, pp. 186-194, 2023. 


\end{thebibliography}


\end{document}
